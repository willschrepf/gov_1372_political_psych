% Options for packages loaded elsewhere
\PassOptionsToPackage{unicode}{hyperref}
\PassOptionsToPackage{hyphens}{url}
\PassOptionsToPackage{dvipsnames,svgnames*,x11names*}{xcolor}
%
\documentclass[
]{article}
\usepackage{lmodern}
\usepackage{amssymb,amsmath}
\usepackage{ifxetex,ifluatex}
\ifnum 0\ifxetex 1\fi\ifluatex 1\fi=0 % if pdftex
  \usepackage[T1]{fontenc}
  \usepackage[utf8]{inputenc}
  \usepackage{textcomp} % provide euro and other symbols
\else % if luatex or xetex
  \usepackage{unicode-math}
  \defaultfontfeatures{Scale=MatchLowercase}
  \defaultfontfeatures[\rmfamily]{Ligatures=TeX,Scale=1}
\fi
% Use upquote if available, for straight quotes in verbatim environments
\IfFileExists{upquote.sty}{\usepackage{upquote}}{}
\IfFileExists{microtype.sty}{% use microtype if available
  \usepackage[]{microtype}
  \UseMicrotypeSet[protrusion]{basicmath} % disable protrusion for tt fonts
}{}
\makeatletter
\@ifundefined{KOMAClassName}{% if non-KOMA class
  \IfFileExists{parskip.sty}{%
    \usepackage{parskip}
  }{% else
    \setlength{\parindent}{0pt}
    \setlength{\parskip}{6pt plus 2pt minus 1pt}}
}{% if KOMA class
  \KOMAoptions{parskip=half}}
\makeatother
\usepackage{xcolor}
\IfFileExists{xurl.sty}{\usepackage{xurl}}{} % add URL line breaks if available
\IfFileExists{bookmark.sty}{\usepackage{bookmark}}{\usepackage{hyperref}}
\hypersetup{
  pdftitle={Gov 1372 - Groups and Identities},
  pdfauthor={your name here},
  colorlinks=true,
  linkcolor=Maroon,
  filecolor=Maroon,
  citecolor=Blue,
  urlcolor=blue,
  pdfcreator={LaTeX via pandoc}}
\urlstyle{same} % disable monospaced font for URLs
\usepackage[margin=1in]{geometry}
\usepackage{longtable,booktabs}
% Correct order of tables after \paragraph or \subparagraph
\usepackage{etoolbox}
\makeatletter
\patchcmd\longtable{\par}{\if@noskipsec\mbox{}\fi\par}{}{}
\makeatother
% Allow footnotes in longtable head/foot
\IfFileExists{footnotehyper.sty}{\usepackage{footnotehyper}}{\usepackage{footnote}}
\makesavenoteenv{longtable}
\usepackage{graphicx,grffile}
\makeatletter
\def\maxwidth{\ifdim\Gin@nat@width>\linewidth\linewidth\else\Gin@nat@width\fi}
\def\maxheight{\ifdim\Gin@nat@height>\textheight\textheight\else\Gin@nat@height\fi}
\makeatother
% Scale images if necessary, so that they will not overflow the page
% margins by default, and it is still possible to overwrite the defaults
% using explicit options in \includegraphics[width, height, ...]{}
\setkeys{Gin}{width=\maxwidth,height=\maxheight,keepaspectratio}
% Set default figure placement to htbp
\makeatletter
\def\fps@figure{htbp}
\makeatother
\setlength{\emergencystretch}{3em} % prevent overfull lines
\providecommand{\tightlist}{%
  \setlength{\itemsep}{0pt}\setlength{\parskip}{0pt}}
\setcounter{secnumdepth}{-\maxdimen} % remove section numbering

\title{Gov 1372 - Groups and Identities}
\author{your name here}
\date{September 23, 2021}

\begin{document}
\maketitle

\hypertarget{marriage-and-partisan-polarization}{%
\section{Marriage and Partisan
Polarization}\label{marriage-and-partisan-polarization}}

Iyengar and Westwood (2014) use answers to questions about a child
marrying an in-party or out-party spouse as one way of characterizing
affective partisan polarization. Some authors have questioned if the way
this question is framed too coarsely. In particular,
\href{https://drive.google.com/file/d/1FOAPqfLQweUFaXtzLGhJT_kkBRwHWkLu/view?usp=sharing}{Klar
et al.~(2018)} argue that, by making the prospective child-in-law's
partisanship salient, the marriage question may be picking up on
respondents dislike of partisanship in general, rather than a dislike of
the opposing party.

The in-class survey you took was a partial replication of the Klar et
al.~(2018) study. We randomized whether you were asked about a
prospective child-in-law who ``frequently talks about politics,''
``rarely talks about politics,'' or a person whose frequency of
discussing politics was not mentioned. This last, control, condition
matches the wording of the question used in Iyengar and Westwood (2014).

\textbf{Data Details:}

\begin{itemize}
\item
  File Name: \texttt{Sep23ClassData\_clean.csv}
\item
  Source: These data are from the survey you took in class. The
  questions are slightly adapted versions of some of the questions used
  in Klar et al (2018) (see
  \href{https://oup.silverchair-cdn.com/oup/backfile/Content_public/Journal/poq/82/2/10.1093_poq_nfy014/1/nfy014_suppl_supplementary_appendix.pdf?Expires=1635436645\&Signature=w4p6NVT1Wrv3tcOJMw~B1LaNxM2-HZIJRtb7fWWqFHKodmLQBO3QeG3qWudEDeJjDT2XhmC3ud8WkNAptT0Hxc3bl47AsAIuMJyQEYMxcJ4W-hYevLRX7GoWNe13yxsXzOe~Q8Fs0kBjiWJf-P9AAkNB-eWZWMtznzPgBnanfRjwWzEB~ziBHcNOGi7I0FNBHSna7Igih6F~tFmuUcSVtvKGJ1IUFE86mc0IyeQLNxzAoz7n7v5ZrI~9J6hzE7wKfDn~ASRJ3icDtx2gN0J2KUVuN-nEIs~0yMNMzp4btpAq8aYp90AmMGHtemsUPoaogAz3PoA4oJNXFCpPNdiFjA__\&Key-Pair-Id=APKAIE5G5CRDK6RD3PGA}{here}
  for the supplemental material of that study with the original
  questionnaire, if you are interested).
\end{itemize}

\begin{longtable}[]{@{}ll@{}}
\toprule
\begin{minipage}[b]{0.33\columnwidth}\raggedright
Variable Name\strut
\end{minipage} & \begin{minipage}[b]{0.61\columnwidth}\raggedright
Variable Description\strut
\end{minipage}\tabularnewline
\midrule
\endhead
\begin{minipage}[t]{0.33\columnwidth}\raggedright
\texttt{pid3}\strut
\end{minipage} & \begin{minipage}[t]{0.61\columnwidth}\raggedright
Political party preference\strut
\end{minipage}\tabularnewline
\begin{minipage}[t]{0.33\columnwidth}\raggedright
\texttt{pid\_lean}\strut
\end{minipage} & \begin{minipage}[t]{0.61\columnwidth}\raggedright
If a respondent didn't identify with the Democrats or Republicans in
\texttt{QID1}, this indicates to which party (or neither) they feel
closer\strut
\end{minipage}\tabularnewline
\begin{minipage}[t]{0.33\columnwidth}\raggedright
\texttt{strongGOP}\strut
\end{minipage} & \begin{minipage}[t]{0.61\columnwidth}\raggedright
Indicator variable for whether the respondent identifies as a Strong
Republican\strut
\end{minipage}\tabularnewline
\begin{minipage}[t]{0.33\columnwidth}\raggedright
\texttt{strongDEM}\strut
\end{minipage} & \begin{minipage}[t]{0.61\columnwidth}\raggedright
Indicator variable for whether the respondent identifies as a Strong
Democrat\strut
\end{minipage}\tabularnewline
\begin{minipage}[t]{0.33\columnwidth}\raggedright
\texttt{strongPARTISAN}\strut
\end{minipage} & \begin{minipage}[t]{0.61\columnwidth}\raggedright
Indicator variable for whether the respondent identifies as a strong
member of either major party\strut
\end{minipage}\tabularnewline
\begin{minipage}[t]{0.33\columnwidth}\raggedright
\texttt{party}\strut
\end{minipage} & \begin{minipage}[t]{0.61\columnwidth}\raggedright
Party variable where those who lean toward either major party are
counted as identifying with that party\strut
\end{minipage}\tabularnewline
\begin{minipage}[t]{0.33\columnwidth}\raggedright
\texttt{treatment}\strut
\end{minipage} & \begin{minipage}[t]{0.61\columnwidth}\raggedright
Which treatment condition the respondent was randomly assigned to\strut
\end{minipage}\tabularnewline
\begin{minipage}[t]{0.33\columnwidth}\raggedright
\texttt{marryDemocrat}\strut
\end{minipage} & \begin{minipage}[t]{0.61\columnwidth}\raggedright
The respondent's answer to how happy they would be if their child
married a Democrat\strut
\end{minipage}\tabularnewline
\begin{minipage}[t]{0.33\columnwidth}\raggedright
\texttt{marryRepublican}\strut
\end{minipage} & \begin{minipage}[t]{0.61\columnwidth}\raggedright
The respondent's answer to how happy they would be if their child
married a Republican\strut
\end{minipage}\tabularnewline
\begin{minipage}[t]{0.33\columnwidth}\raggedright
\texttt{inPartyHappy}\strut
\end{minipage} & \begin{minipage}[t]{0.61\columnwidth}\raggedright
Indicator variable for whether the respondent would be happy if their
child married a member of their own party\strut
\end{minipage}\tabularnewline
\begin{minipage}[t]{0.33\columnwidth}\raggedright
\texttt{outPartyUnhappy}\strut
\end{minipage} & \begin{minipage}[t]{0.61\columnwidth}\raggedright
Indicator variable for whether the respondent would be unhappy if their
child married a member of the other major party\strut
\end{minipage}\tabularnewline
\begin{minipage}[t]{0.33\columnwidth}\raggedright
\texttt{polarized}\strut
\end{minipage} & \begin{minipage}[t]{0.61\columnwidth}\raggedright
Indicator variable for whether the respondent was affectively
polarized\strut
\end{minipage}\tabularnewline
\bottomrule
\end{longtable}

Once again, the .Rmd version of this file has code you can use to load
the data.

These data are \emph{not} the raw output from the survey you took. In
particular, all of the indicator variables are the result of coding
decisions and data processing done by the instructors (based on the
procedures used in Klar et al.~(2018)). For the first few questions,
just open up the data and take a look at it (ask us if you need help
viewing the data in spreadsheet format in RStudio).

\hypertarget{question-1}{%
\subsection{Question 1}\label{question-1}}

\textbf{How were the \texttt{inPartyHappy} and \texttt{outPartyUnhappy}
variables defined? Does this seem like a reasonable procedure? Do you
notice any missing values? Why are they missing? How might the missing
data affect researchers' inferences?}

\hypertarget{question-2}{%
\subsection{Question 2}\label{question-2}}

\textbf{How was the \texttt{polarized} variable defined? Is there
another way you might consider coding this variable for individual
polarization? What would be an advantage and a disadvantage of your
alternative approach?}

Now let's take a look at if there are differences in some of the key
outcome variables depending on treatment status. Here is an example of
how you might make a graph to look at if the rates of unhappiness with a
prospective out-party in-law differ depending on the frequency with
which they talk about politics.

\includegraphics{data_assignment3_sept_23_2021_files/figure-latex/original_plot-1.pdf}

\hypertarget{question-3}{%
\subsection{Question 3}\label{question-3}}

\textbf{Comment on what you see in the example graph. Did the treatment
affect unhappiness with a prospective out-party member coming into the
family?}

Yes--there appears to be a slight increase in outparty unhappiness for
the `frequently' group and a slight decrease in outparty happiness for
the `rarely' group relative to the control.

\hypertarget{question-4}{%
\subsection{Question 4}\label{question-4}}

\textbf{Did the different treatment conditions affect the proportions of
people who were affectively polarized? Make a plot and explain what you
found.}

\begin{verbatim}
## # A tibble: 89 x 2
##    treatment  polarized
##    <chr>      <lgl>    
##  1 Control    TRUE     
##  2 Control    TRUE     
##  3 Control    TRUE     
##  4 Frequently FALSE    
##  5 Frequently TRUE     
##  6 Rarely     FALSE    
##  7 Control    TRUE     
##  8 Rarely     NA       
##  9 Frequently TRUE     
## 10 Control    TRUE     
## # ... with 79 more rows
\end{verbatim}

\hypertarget{question-5}{%
\subsection{Question 5}\label{question-5}}

\textbf{Take a quick look at Figure 1 and Figure 2 in
\href{https://academic.oup.com/poq/article-abstract/82/2/379/4996003}{Klar
et al.~(2018)}. How do the results from our in-class data compare to
their results? What might explain any differences? If there aren't an
notable differences, is there a common pattern across the two datasets
that is puzzling? What hypothesis do you have to explain it.}

\hypertarget{question-6-data-science-question}{%
\subsection{Question 6 (Data Science
Question)}\label{question-6-data-science-question}}

\textbf{We might also be interested in if things looked different for
weak vs.~strong partisans. Pick one of the two outcome variables you
just examined and make a plot that would help us understand if responses
within and across treatment groups differ between weak and strong
partisans.}

\includegraphics{data_assignment3_sept_23_2021_files/figure-latex/q6-1.pdf}

\hypertarget{question-7}{%
\subsection{Question 7}\label{question-7}}

\textbf{Are there any other issues you can think of that might confound
the utility of the marriage question as a measure of affective
polarization? If you have any concerns, how might you design a study to
evaluate your hypotheses?}

\hypertarget{question-8}{%
\subsection{Question 8}\label{question-8}}

\textbf{Based on the data and your work on this assignment, are there
any changes you would make to the Iyengar and Westwood (2014) study or
the Klar et al.~(2018) study or extensions of the research that you
would like to see? (For example, would you alter the wording of any
questions, change the experimental protocol, or come to any different
conclusions?)}

\end{document}
